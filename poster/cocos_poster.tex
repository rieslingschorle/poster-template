\documentclass[
  %draft,
  a0paper,
  portrait,
  fontscale=.35 % scales inversely (larger value=smaller font). Default is 0.292. Also affects Title etc!
  ]{baposter} 


\usepackage{tikz}
\usepackage{graphicx}

\usetikzlibrary{calc}
\usetikzlibrary{svg.path}

% use PT sans font

\usepackage{textcomp}
\usepackage[T1]{fontenc}
\usepackage[no-math]{fontspec}		%Font im Mathmode wird weiter unten gesetzt
\setmainfont{Red Hat Text}				%Uni-Schriftart
\setsansfont{Red Hat Text}

\usepackage[utf8]{inputenc}			%Umlaute

\usepackage{amsmath}				%Mehr Mathe-Funktionen
\usepackage{mathspec}				%Mehr Kontrolle über Mathmode-Formatierungen
\setmathsfont(Digits){Red Hat Text}		%PT Sans im Mathmode\phantom{sp} 
\setmathsfont(Latin){Red Hat Text}
\setmathrm{Red Hat Text}

\usepackage{xfrac}					%schöne, diagonale Bruchstriche
\usepackage{braket}					%Bras und Kets
\usepackage{derivative}

\usepackage{graphbox}				%enable centered alignment of image to text

% Global Color Definitions
%

\definecolor{rptublaugrau}{cmyk}{.70, .44, .30, .15}
\definecolor{rptugruengrau}{cmyk}{.55, .10, .30, 0}
\definecolor{rptudunkelblau}{cmyk}{1.00, .85, .40, .30}
\definecolor{rptuhellblau}{cmyk}{.58, .11, 0, 0}
\definecolor{rptudunkelgruen}{cmyk}{.85, .30, .50, .25}
\definecolor{rptuhellgruen}{cmyk}{.56, 0, .58, 0}
\definecolor{rptuviolett}{cmyk}{.85, .90, .20, 0.08}
\definecolor{rptupink}{cmyk}{.10, .90, 0, 0}
\definecolor{rpturot}{cmyk}{0, .95, .55, 0}
\definecolor{rptuorange}{cmyk}{0, .45, .70, 0}
\definecolor{rptuschwarz}{cmyk}{0, 0, 0, 1.00}
\definecolor{rptuweiss}{cmyk}{0, 0, 0, 0}

\colorlet{schiefer}{rptublaugrau}
\colorlet{ozean}{rptugruengrau}
\colorlet{nacht}{rptudunkelblau}
\colorlet{tag}{rptuhellblau}
\colorlet{petrol}{rptudunkelgruen}
\colorlet{apfel}{rptuhellgruen}
\colorlet{pflaume}{rptuviolett}
\colorlet{fuchsia}{rptupink}
\colorlet{himbeere}{rpturot}
\colorlet{mango}{rptuorange}

\let\OLDitemize\itemize
\renewcommand\itemize{\OLDitemize\setlength{\itemsep}{0.1em}}

%choose your poster colors here:
%todo choose pair color automatically
\colorlet{color_background}{mango}
\colorlet{color_accents}{himbeere}


\tikzstyle{logobox} = [draw=rptuweiss, fill=rptuweiss, thin,
    rectangle, inner sep=10pt, inner ysep=20pt, minimum width=10cm,
                        minimum height = 3cm]

\tikzstyle{titelbox} = [draw=rptuweiss, fill=rptuweiss, thin,
    rectangle, inner sep=10pt, inner ysep=20pt, minimum width=20cm,
                        minimum height = 3cm]

\begin{document}

\background{
  \begin{tikzpicture}[remember picture,overlay]%
%carshitbrown] ([yshift=5em] current page.south west) rectangle ([yshift=4em]current page.south east);
    %the header
    \fill [fill=color_background] (current page.north west) rectangle (current page.south east);
    %\node [logobox] (logobox) at ($(current page.north west)+(4cm,-2.25cm)$){};
    %\node [titelbox] (titelbox) at ($(current page.north west)+(+19.5cm,-2.25cm)$){};
  \end{tikzpicture}
}

\begin{poster}{
    %general options for the poster
    %grid=true,
    grid=false,
    columns=3,
    %  colspacing=4.2mm,
    headerheight=0.1\textheight,
    background=user,
    eyecatcher=true,
    boxpadding=2em,
    %posterbox options
    headerborder=open,
    borderColor=color_background,
    headershape=rectangle,
    headershade=plain,
    %headerColorOne=mango!50,
    headerColorOne=rptuweiss,
    %  headerColortwo=yellow!42, %is used when the header background is shaded
    textborder=roundedright,
    boxshade=plain,
    boxColorOne=rptuweiss,
    %  boxColorTwo=cyan!42,%is used when the text background is shaded
    headerFontColor=color_accents,
    headerfont=\Large\sf\bf,
    linewidth=1pt
  }
  {
  \input{header}
  }
  %the poster title
  {%
  \color{rptuweiss}\bf \LARGE%
  Something with mathematics
  }
  %the author(s)
  {\color{rptuweiss}\large%
	Coco, Coco Loco, Corinna %
  }
  %the logo (the logo on the top right)
  {
    
  }
  
  \headerbox{}%
  {name=foot, column=0, span=3, above=bottom,textborder=none,headerborder=none,boxheaderheight=0pt, boxshade=none}
  {\hfill \vspace{0.5em}  \hfill \color{white} \huge www.coco.loco \hfill \hfill \raisebox{-0.5em}{} \hspace{0.75em} \raisebox{-0.25em}{} \vspace{-0.5em}}
  
  %====================================================================================%  
% Introduction box spanning all columns
\begin{posterbox}[name=intro,column=0,row=0, span=3]{Introduction}
	\begin{center}
		\includegraphics[width=0.5\textwidth]{imgs/coco_loco.jpg}
	\end{center}
\end{posterbox}

%====================================================================================%  
% left
\begin{posterbox}[name=assumption,column=0,row=1, below=intro]{Assumption}
	\textbf{Some mathematics stuff}
	\vspace{4cm}
	\begin{center}
		\includegraphics[width=\textwidth]{imgs/coco_affe.jpg}
	\end{center}
	\vspace{5cm}
\end{posterbox}


%====================================================================================%  
% center
\begin{posterbox}[name=proof,column=1,row=1, below=intro]{Proof}
	\textbf{Something something in the month of May}
	\vspace{5cm}
\end{posterbox}

\begin{posterbox}[name=imgs,column=1,row=1, below=proof, bottomaligned=assumption]{Nice pictures}
	\textbf{This box is bottom aligned with "assumptions"}
\end{posterbox}

%====================================================================================%  
% right
\begin{posterbox}[name=alc,column=2,row=1, below=intro]{I don't know}
	\textbf{Schorle?}
	\begin{itemize}
		\item Warum gibts nur schöne Bilder mit rosé Schorle?
	\end{itemize}
	\includegraphics[width=\textwidth]{imgs/schorle.jpg}
	\vspace{3cm}
\end{posterbox}

\begin{posterbox}[name=rating ,column=2,row=1, below=alc, bottomaligned=assumption]{Rating}
	\begin{center}
		\includegraphics[width=0.5\textwidth]{imgs/10_10.jpeg}
	\end{center}
\end{posterbox}


\end{poster}
\end{document}
